%%%%%%%%%%%%%%%%%%%%%%%%%%%%%%%%%%%%%%%%%
% Academic CV - Carlos Denner dos Santos (FRENCH VERSION)
% Based on Medium Length Professional CV Template
% NOTE: This is the French version. Keep in sync with CarlosDenner_CV_resume.tex (English)
% See CV_SYNC_GUIDE.md for synchronization instructions
%%%%%%%%%%%%%%%%%%%%%%%%%%%%%%%%%%%%%%%%%

\documentclass[
	11pt, % Font size
]{resume} % Use the resume class

\usepackage{ebgaramond} % Use the EB Garamond font
\usepackage{hyperref} % For clickable links
\usepackage[french]{babel} % French language support

% Hyperref setup
\hypersetup{
    colorlinks=true,
    linkcolor=black,
    urlcolor=blue,
    citecolor=black
}

% Bibliography setup
\usepackage[
    backend=biber,
    style=numeric,
    sorting=ydnt,
    maxbibnames=99,
    giveninits=true,
    defernumbers=true,
    refsection=section,
]{biblatex}

\addbibresource{publications.bib}

% Custom bibliography formatting
% Title formatting: bold and clickable if DOI exists
\DeclareFieldFormat{title}{%
  \iffieldundef{doi}
    {\textbf{#1}}
    {\href{https://doi.org/\thefield{doi}}{\textbf{#1}}}}
\DeclareFieldFormat[article]{title}{%
  \iffieldundef{doi}
    {\textbf{#1}}
    {\href{https://doi.org/\thefield{doi}}{\textbf{#1}}}}
\DeclareFieldFormat[incollection]{title}{%
  \iffieldundef{doi}
    {\textbf{#1}}
    {\href{https://doi.org/\thefield{doi}}{\textbf{#1}}}}
\DeclareFieldFormat[misc]{title}{%
  \iffieldundef{doi}
    {\textbf{#1}}
    {\href{https://doi.org/\thefield{doi}}{\textbf{#1}}}}

\renewbibmacro{in:}{}
\DeclareFieldFormat{journaltitle}{\textit{#1}}
\DeclareFieldFormat[article]{volume}{\textbf{#1}}
\setlength\bibitemsep{0.5\baselineskip}

% Hide DOI/URL display (titles are clickable instead)
\DeclareFieldFormat{doi}{}
\DeclareFieldFormat{url}{}

%------------------------------------------------

\name{Carlos Denner dos Santos} % Your name

\address{Spécialiste en IA \\ Montréal, QC, Canada}

\address{
    \href{mailto:carlosdenner@gmail.com}{carlosdenner@gmail.com} \\
    +1 (438) 836-4116
}

\address{
    \href{https://www.linkedin.com/in/carlosdenner}{LinkedIn} $\diamond$ 
    \href{https://www.researchgate.net/profile/Carlos-Santos-Jr}{ResearchGate} \\
    \href{https://scholar.google.com/citations?user=LZQKjOEAAAAJ}{Google Scholar}
}

%----------------------------------------------------------------------------------------

\begin{document}

%----------------------------------------------------------------------------------------
%	DOMAINES DE RECHERCHE
%----------------------------------------------------------------------------------------

\begin{rSection}{Domaines de Recherche}
	Sécurité des LLM et défense contre l'injection de prompts, IA affective et interaction humain-LLM, analyse des sentiments et sentiments algorithmiques, gouvernance de l'IA et systèmes éthiques, génération augmentée par récupération (RAG) et atténuation des hallucinations, théorie de l'innovation numérique et contrôle réversible, inclusion numérique pilotée par l'IA, écosystèmes de logiciels libres, gouvernance informatique et transformation numérique.
\end{rSection}

%----------------------------------------------------------------------------------------
%	FORMATION
%----------------------------------------------------------------------------------------

\begin{rSection}{Formation}
	
	\textbf{University of Nottingham, Angleterre} \hfill \textit{Décembre 2010 - Décembre 2011} \\ 
	PostDoc en Informatique \\
	Superviseur : George Kuk
	
	\textbf{Université de São Paulo, Brésil} \hfill \textit{Août 2009 - Juillet 2011} \\ 
	PostDoc en Informatique \\
	Superviseur : Fabio Kon
	
	\textbf{Southern Illinois University, États-Unis} \hfill \textit{Août 2005 - Juillet 2009} \\ 
	Doctorat en Systèmes d'Information \\
	Superviseur : John Pearson
	
	\textbf{Université Fédérale de Minas Gerais, Brésil} \hfill \textit{Février 2003 - Février 2005} \\ 
	Maîtrise en Gestion
	
	\textbf{Université de l'État de Minas Gerais, Montes Claros} \hfill \textit{Février 1999 - Décembre 2002} \\ 
	Baccalauréat en Commerce
	
	\textbf{École Technique de Montes Claros, Brésil} \hfill \textit{Février 1997 - Décembre 1999} \\ 
	Diplôme Technique en Traitement des Données
	
\end{rSection}

%----------------------------------------------------------------------------------------
%	EXPÉRIENCE PROFESSIONNELLE
%----------------------------------------------------------------------------------------

\begin{rSection}{Expérience Professionnelle (1999--2025)}

	\textbf{Expert en IA} \hfill \textit{Juillet 2025 - Présent} \\
	\href{https://videns.ai}{Videns AI} \hfill Montréal, Canada
	
	\textbf{Consultant en Science des Données} \hfill \textit{Novembre 2021 - Janvier 2025} \\
	Bell Canada \hfill Montréal, Canada
	
	\textbf{Chercheur Associé} \hfill \textit{Mai 2021 - Avril 2023} \\
	École de Technologie Supérieure \hfill Montréal, Canada
	
	\textbf{Co-Chercheur} \hfill \textit{Février 2020 - Présent} \\
	\href{https://jooay.com/}{Application Jooay}, Université McGill \hfill Montréal, Canada
	
	\textbf{Assistant de Recherche} \hfill \textit{Novembre 2019 - Février 2020} \\
	Centre de Recherche CHU Sainte-Justine \hfill Montréal, Canada
	
	\textbf{Chercheur Associé} \hfill \textit{Juin 2019 - Mai 2021} \\
	Université du Québec à Montréal (UQAM) \hfill Montréal, Canada
	
	\textbf{Membre du Conseil d'Administration} \hfill \textit{Juin 2017 - Décembre 2018} \\
	FINATEC (Fondation des Projets Technologiques) \hfill Brasília, Brésil
	
	\textbf{Consultant} \hfill \textit{Novembre 2016 - Février 2017} \\
	Tribunal de Contas da União (Cour des Comptes Fédérale) \hfill Brasília, Brésil
	
	\textbf{Directeur Adjoint} \hfill \textit{Juin 2016 - Décembre 2020} \\
	Programme de Doctorat en Gestion, Université de Brasília \hfill Brasília, Brésil
	
	\textbf{Directeur Associé} \hfill \textit{Janvier 2016 - Décembre 2017} \\
	Centre de Développement Technologique (CDT), Université de Brasília \hfill Brasília, Brésil
	
	\textbf{Consultant (Gouvernance des Logiciels Libres)} \hfill \textit{Février 2014 - Juin 2016} \\
	Ministère de la Planification, Gouvernement du Brésil \hfill Brésil
	
	\textbf{Consultant} \hfill \textit{Novembre 2013 - Février 2014} \\
	Ministère de la Planification, Gouvernement du Brésil \hfill Brésil
	
	\textbf{Fondateur et Directeur} \hfill \textit{Janvier 2013 - Décembre 2023} \\
	Laboratoire de Recherche Socie-Dados, Université de Brasília \hfill Brasília, Brésil
	
	\textbf{Professeur Associé} \hfill \textit{Décembre 2011 - Décembre 2023} \\
	Université de Brasília \hfill Brasília, Brésil
	
	\textbf{Chercheur Postdoctoral} \hfill \textit{Décembre 2010 - Décembre 2011} \\
	Université de Nottingham \hfill Nottingham, Royaume-Uni
	
	\textbf{Professeur Collaborateur} \hfill \textit{Août 2010 - Janvier 2011} \\
	Université de São Paulo \hfill São Paulo, Brésil
	
	\textbf{Chercheur Postdoctoral} \hfill \textit{Août 2009 - Août 2011} \\
	Université de São Paulo \hfill São Paulo, Brésil
	
	\textbf{Conseiller en Projet (SoftwarePublico.gov.br)} \hfill \textit{Juillet 2009 - Décembre 2010} \\
	CTI Renato Archer \hfill Campinas, Brésil
	
	\textbf{Instructeur Diplômé} \hfill \textit{Janvier 2008 - Mai 2008} \\
	Southern Illinois University Carbondale \hfill Carbondale, IL, États-Unis
	
	\textbf{Instructeur (Formation Professionnelle)} \hfill \textit{Juin 2005} \\
	FAPEMIG, Fondation de Recherche de Minas Gerais \hfill Brésil
	
	\textbf{Chargé de Cours à Temps Partiel} \hfill \textit{Août 2004 - Juillet 2005} \\
	Universidade FUMEC \hfill Belo Horizonte, Brésil
	
	\textbf{Consultant (TI)} \hfill \textit{Mai 2004 - Décembre 2004} \\
	Bioquímica e Química do Brasil Ltda. \hfill Belo Horizonte, Brésil
	
	\textbf{Professeur Suppléant} \hfill \textit{Mars 2004 - Août 2005} \\
	Université Fédérale de Minas Gerais \hfill Belo Horizonte, Brésil
	
	\textbf{Programmeur Informatique} \hfill \textit{Août 2002 - Février 2003} \\
	Universidade Estadual de Montes Claros \hfill Montes Claros, Brésil
	
	\textbf{Professeur} \hfill \textit{Août 2001 - Février 2003} \\
	Fondation Éducationnelle Montes Claros \hfill Montes Claros, Brésil
	
	\textbf{Administrateur Réseau / Programmeur Informatique} \hfill \textit{Mars 1999 - Janvier 2003} \\
	Unimed Montes Claros \hfill Montes Claros, Brésil

\end{rSection}

%----------------------------------------------------------------------------------------
%	ENCADREMENTS ACADÉMIQUES
%----------------------------------------------------------------------------------------

\begin{rSection}{Encadrements Académiques}

	\subsection*{Encadrements Postdoctoraux}
	
	\textbf{Pedro Jacome de Moura Junior} \hfill \textit{2023} \\
	Universidade de Brasília
	
	\textbf{Almir Oliveira Junior} \hfill \textit{2018} \\
	Instituto de Pesquisa Econômica Aplicada (IPEA), Universidade de Brasília
	
	\textbf{Fabio Buiati} \hfill \textit{2017} \\
	Centro de Apoio ao Desenvolvimento Tecnológico, Universidade de Brasília
	
	\subsection*{Encadrements de Doctorat}
	
	\textbf{Anna Carolina Ribeiro} \hfill \textit{2022} \\
	Gestion des risques en administration publique $\cdot$ Universidade de Brasília
	
	\textbf{Claudia Tolentino Santos} \hfill \textit{2021} \\
	Gouvernance des pensions et performance financière $\cdot$ Universidade de Brasília
	
	\textbf{Pablo Péron} \hfill \textit{2021} \\
	Survie organisationnelle post-incubation (analyse pilotée par l'IA) $\cdot$ Universidade de Brasília
	
	\textbf{Gustavo Alves} \hfill \textit{2021} \\
	Gouvernance informatique dans le secteur public $\cdot$ Universidade de Brasília
	
	\textbf{Patrícia Almeida} \hfill \textit{2019--2024} \\
	Gouvernance de l'IA et gouvernement numérique $\cdot$ Universidade de Brasília
	
	\textbf{Deise Goulart} \hfill \textit{2021--Présent} \\
	Efficacité des ONG et du secteur public $\cdot$ Universidade de Brasília
	
	\textbf{Isabela Ferraz} \hfill \textit{2019} \\
	Gouvernance des communautés virtuelles $\cdot$ Universidade de Brasília
	
	\textbf{Silvia Satiko Onoyama} \hfill \textit{2018} \\
	Mécanismes pour une gouvernance informatique effective $\cdot$ Universidade de Brasília
	
	\textbf{Paulo Meirelles} (co-superviseur) \hfill \textit{2013} \\
	Surveillance des métriques de logiciels libres $\cdot$ Universidade de São Paulo
	
	\subsection*{Encadrements de Maîtrise}
	
	\textbf{Leonel Cerqueira Santos} \hfill \textit{2016} \\
	Mécanismes non-opérationnels dans l'efficacité de la gouvernance informatique $\cdot$ Universidade de Brasília
	
	\textbf{Daniel Shim de Sousa Esashika} \hfill \textit{2016} \\
	Influence des parrains sur les structures des communautés de logiciels libres $\cdot$ Universidade de Brasília
	
	\textbf{Isadora Vergara} \hfill \textit{2015} \\
	Gestion adaptative vs. déterminisme environnemental dans les blogs $\cdot$ Universidade de Brasília
	
	\textbf{Leonardo Oliveira} \hfill \textit{2015} \\
	Conséquences de l'adoption de l'innovation $\cdot$ Universidade de Brasília
	
	\textbf{Juliana Miranda} \hfill \textit{2015} \\
	Variables environnementales et organisationnelles dans la performance des startups $\cdot$ Universidade de Brasília
	
	\textbf{Luiz Fernando Silva Pinto} \hfill \textit{2015} \\
	Facteurs influençant l'effort et les contributions sur Wikipedia $\cdot$ Universidade de Brasília
	
	\subsection*{Encadrements Sélectionnés de Recherche de Premier Cycle}
	
	2020--2021 : Thiago Lopes, Sara Andrade, Rayssa Lorrane, Rafael Azevedo Lima, Eduardo Martins, Edilson Niehues Rodrigues Lima \\
	2018--2019 : Valesca Scarlat C. da Fonseca, Raissa Paiva Pires, Vitor Gabriel G. da Silva, Saulo Barros de Melo, Ateldy Brasil Filho \\
	2012--2014 : Leonardo Alves dos Santos, André S. R. de Souza Marques, Wilton da Silva Rodrigues, Thiago F. Figueiredo, Lucas B. Cusinato Rodrigues \\
	2009--2010 : Rafael S. Suguiura (USP), Marcos Bonci (USP)
	
	\subsection*{Projets Sélectionnés de Spécialisation/MBA et Projets de Fin d'Études de Premier Cycle}
	
	2021 : Luiz Felipe Pimenta de Araujo (Jeux sérieux dans l'enseignement de la stratégie d'affaires) \\
	2014--2016 : 15+ encadrements dans des domaines de stratégie d'affaires et de gestion

\end{rSection}

%----------------------------------------------------------------------------------------
%	PUBLICATIONS
%----------------------------------------------------------------------------------------

\begin{rSection}{Publications}

\nocite{*} % Include all entries from publications.bib

\subsection*{Articles de Revue}
\begin{refsection}
\nocite{*}
\printbibliography[heading=none, type=article, notkeyword=review, notkeyword=dataset]
\end{refsection}

\subsection*{Articles de Conférence}
\begin{refsection}
\nocite{*}
\printbibliography[heading=none, type=inproceedings]
\end{refsection}

\subsection*{Chapitres de Livre}
\begin{refsection}
\nocite{*}
\printbibliography[heading=none, type=incollection]
\end{refsection}

\subsection*{Ensembles de Données}
\begin{refsection}
\nocite{*}
\printbibliography[heading=none, type=misc, keyword=dataset]
\end{refsection}

\subsection*{Examens Évaluation Ouverts}
\begin{refsection}
\nocite{*}
\printbibliography[heading=none, type=article, keyword=review]
\end{refsection}

\subsection*{En Révision et En Préparation}

[1] \textbf{Denner, C. D.} (et al.). Building an LLM Firewall: A Multi-Phase Defense Against Prompt Injection -- From Patent Landscape to Deployable Input-Side Guardrails. \textit{Communications of the ACM} -- En révision.

[2] \textbf{Denner, C. D.} (et al.). Reversible Control as a Digital Innovation Theory: Changing the Control--Learning Relationship through Knowledge Deployment Capability. \textit{MISQ Theory \& Review track} -- Manuscrit en préparation.

[3] \textbf{Denner, C. D.} LLMs, Sentiment Analysis, and Algorithmic Feelings: Toward a Theory of Affective Loops in Human--LLM Interaction. \textit{Academy of Management Review} -- Manuscrit en préparation.

[4] \textbf{Denner, C. D.} (et al.). The AI Recommendation System of Jooay.com: Enhancing Digital Inclusion for Children and Youth with Disabilities. Manuscrit en préparation.

[5] \textbf{Denner, C. D.} (et al.). Evaluating and Mitigating Hallucinations in Retrieval-Augmented Generation (RAG): An Experimental Framework. Manuscrit en préparation.

\end{rSection}

%----------------------------------------------------------------------------------------
%	SUBVENTIONS DE RECHERCHE
%----------------------------------------------------------------------------------------

\begin{rSection}{Subventions de Recherche}

	\textbf{FAP-DF Demanda Espontânea} \hfill \textit{2016--2021} \\
	Chercheur Principal \\
	Fondation de Soutien à la Recherche du Distrito Fédéral \\
	
	\textbf{CAPES Pesquisador Júnior} \hfill \textit{2018} \\
	Chercheur Principal \\
	Subvention pour Jeune Chercheur, Coord. pour l'Amélioration du Personnel de l'Enseignement Supérieur \\
	
	\textbf{CNPq Edital Universal} \hfill \textit{2013--2015} \\
	Chercheur Principal \\
	Subvention de Recherche Universelle, Conseil National pour le Développement Scientifique et Technologique \\

\end{rSection}

%----------------------------------------------------------------------------------------
%	PRIX ET DISTINCTIONS
%----------------------------------------------------------------------------------------

\begin{rSection}{Prix et Distinctions}

	\textbf{Bourse Postdoctorale de l'Institut Horizon} \hfill \textit{2011--2012} \\
	Institut de Recherche Horizon Digital Economy, Royaume-Uni $\cdot$ Université de Nottingham
	
	\textbf{Bourse Postdoctorale FAPESP} \hfill \textit{2009--2010} \\
	Fondation de Recherche de São Paulo, Brésil $\cdot$ Université de São Paulo
	
	\textbf{Bourse du Centre de Recherche Pontikes} \hfill \textit{2007} \\
	Southern Illinois University $\cdot$ Étude de partenariat communautaire de logiciels libres
	
	\textbf{Bourse Commémorative John M. Fohr} \hfill \textit{2007} \\
	Fondation Southern Illinois University $\cdot$ Excellence dans les études de gestion
	
	\textbf{Bourse Fulbright/CAPES} \hfill \textit{2005--2009} \\
	Commission Fulbright \& CAPES, Brésil $\cdot$ Bourse complète pour études de doctorat aux États-Unis
	
	\textbf{Bourse de Maîtrise CNPq} \hfill \textit{2003--2005} \\
	CNPq, Brésil $\cdot$ Bourse d'études pour M.Sc. en Gestion
	
	\textbf{Mentions Honorables} \hfill \textit{2013, 2015} \\
	Deux articles de recherche ont reçu une mention honorable à EnANPAD (conférence nationale brésilienne)

\end{rSection}

%----------------------------------------------------------------------------------------
%	SERVICE PROFESSIONNEL
%----------------------------------------------------------------------------------------

\begin{rSection}{Service Professionnel}

	\textbf{Membre du Comité Éditorial} \hfill \textit{2015--2025} \\
	Journal of Global Information Technology Management
	
	\textbf{Examinateur ad-hoc (Organismes de Financement)} \hfill \textit{2015--Présent} \\
	Propositions de subvention de recherche pour CAPES (2015) et FAP-DF (2015--Présent)
	
	\textbf{Membre du Conseil d'Administration} \hfill \textit{2017--2018} \\
	Fondation des Projets Technologiques (FINATEC), Brasília $\cdot$ Partenariats universités-industrie
	
	\textbf{Consultant (Analytique des Données)} \hfill \textit{2016--2017} \\
	Tribunal de Contas da União (Cour d'Audit Fédérale), Brésil $\cdot$ Analyse des données éducatives
	
	\textbf{Consultant (Gouvernance des Logiciels Libres)} \hfill \textit{2014--2016} \\
	Ministère de la Planification -- softwarepublico.gov.br $\cdot$ Gouvernance de plateforme et licences
	
	\textbf{Consultant (Gestion Informatique)} \hfill \textit{2013--2014} \\
	Ministère de la Planification, Brésil $\cdot$ Estimation de la main-d'œuvre informatique et soutien décisionnel
	
	\textbf{Membre du Comité d'Examen de Doctorat} \hfill \textit{2013--2023} \\
	Multiples comités de dissertations à Universidade de Brasília et autres institutions
	
	\textbf{Membre du Comité d'Embauche Académique} \hfill \textit{Années Diverses} \\
	Panels d'embauche de professeurs universitaires et comités de sélection de bourses d'études supérieures
	
	\textbf{Examinateur (14 examens, 2021--2025)} \\
	$\cdot$ Future Business Journal (2025) \\
	$\cdot$ Heliyon -- Cell Press (2024) \\
	$\cdot$ Public Management Review (2023, 2 examens) \\
	$\cdot$ Journal of Global Information Technology Management (2022, 2023, 3 examens) \\
	$\cdot$ Journal of Open Innovation: Technology, Market, and Complexity (2023) \\
	$\cdot$ Journal of Forensic Psychology Research and Practice (2023) \\
	$\cdot$ IEEE Access (2022, 2 examens) \\
	$\cdot$ ACM Transactions on Software Engineering and Methodology (TOSEM) (2022) \\
	$\cdot$ Revista de Administração Contemporânea (2022)
	
	\textbf{Examinateur de Conférence} \\
	AMCIS, ECIS, ICIS, AoM, ANPAD (EnANPAD), et autres conférences en systèmes d'information et gestion

\end{rSection}

%----------------------------------------------------------------------------------------
%	ADHÉSIONS PROFESSIONNELLES
%----------------------------------------------------------------------------------------

\begin{rSection}{Adhésions Professionnelles}

	\textbf{Association for Information Systems (AIS)} \hfill \textit{2015 - Présent}
	
	\textbf{Académie Brésilienne de Gestion (ANPAD)} \hfill \textit{2010 - Présent}
	
	\textbf{Academy of Management (AOM)} \hfill \textit{2016 - Présent}

\end{rSection}

%----------------------------------------------------------------------------------------

\end{document}

