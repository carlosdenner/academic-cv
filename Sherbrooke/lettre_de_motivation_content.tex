Sherbrooke, le 17 novembre 2025

Objet : Candidature au poste de professeure ou professeur en gestion de
l'intelligence artificielle (offre 07823)

Madame, Monsieur,

Je vous soumets ma candidature au poste de professeure ou professeur en
gestion de l'intelligence artificielle (offre 07823) au Département des
systèmes d'information et méthodes quantitatives de gestion (SIMQG) de
l'École de gestion de l'Université de Sherbrooke.

Titulaire d'un doctorat en systèmes d'information au sein d'une faculté
de gestion, complété par des stages postdoctoraux en informatique et en
statistique, je travaille depuis plus de vingt ans à l'intersection de
l'analytique, des systèmes d'information et de la stratégie. Mes projets
récents portent sur la gouvernance et la régulation de l'IA, la gestion
de portefeuilles de projets d'IA, ainsi que sur l'ingénierie et
l'exploitation de systèmes d'IA à grande échelle, incluant des modèles
de type LLM et des systèmes de recommandation.

Sur le plan scientifique, j'ai publié notamment dans Ethics and
Information Technology (« Artificial Intelligence Regulation: a
framework for governance ») et dans Government Information Quarterly («
Artificial intelligence governance: Understanding how public
organizations implement it »). Ces travaux proposent, d'une part, un
cadre intégrateur pour la régulation de l'IA, et d'autre part, une étude
empirique de la mise en œuvre de la gouvernance de l'IA dans 28
organisations publiques sur cinq continents. Ils s'inscrivent
directement dans les thématiques du poste, en particulier la gouvernance
et la gestion des risques de l'IA ainsi que la sécurité et la résilience
des systèmes d'IA.

Parallèlement, j'ai conçu et déployé des systèmes analytiques et d'IA
dans des secteurs variés (télécommunications, énergie, santé, économie
sociale), ce qui me donne une vision très concrète des enjeux
d'AIOps/MLOps et de cycle de vie de l'IA en contexte organisationnel.
Mes recherches actuelles portent sur la sécurité des LLM (défense contre
la prompt injection, détection des hallucinations dans les systèmes
RAG), l'intelligence artificielle affective et les boucles affectives
dans l'interaction humain-LLM, ainsi que les théories de l'innovation
numérique (notamment le contrôle réversible). Ces travaux se traduisent
par des manuscrits pour MISQ, Academy of Management Review et
Communications of the ACM. J'ai récemment dirigé un projet de
cartographie de plus de 250 000 familles de brevets en apprentissage
automatique (2010--2025) afin d'identifier les domaines émergents liés à
la détection des hallucinations, à la défense contre la prompt
injection, à la sécurité des agents et à la fédéralisation de
l'entraînement de LLM. Ce travail illustre ma manière de faire : croiser
rigueur analytique, compréhension fine des organisations et
préoccupations stratégiques.

Le programme de recherche que je propose s'articule autour de trois axes
: (1) la gouvernance et la gestion des risques des systèmes d'IA dans
les organisations ; (2) l'ingénierie, les opérations et la sécurité des
systèmes d'IA (AIOps/MLOps) avec un focus sur la sécurité des LLM
(prompt injection, hallucinations), les systèmes agentiques et l'IA
affective ; et (3) la performance, l'impact et la durabilité de l'IA en
contexte organisationnel. L'objectif est de produire à la fois des
contributions théoriques (théories de l'innovation numérique, modèles de
gouvernance, cadres d'évaluation de la sécurité des LLM, théories des
boucles affectives) et des outils directement utiles aux organisations
(référentiels de défense contre la prompt injection, patrons
d'architecture pour systèmes RAG sécurisés, tableaux de bord, cas
d'enseignement), en forte synergie avec le SIMQG, le Centre de recherche
Createch sur les organisations intelligentes (CROI) et le Centre Laurent
Beaudoin.

En enseignement, j'ai une expérience significative à tous les cycles
dans des cours de systèmes d'information, d'analytique d'affaires, de
transformation numérique et de gouvernance de l'IA. Ma philosophie est
de partir de problèmes réels, d'outiller les étudiantes et étudiants
pour les structurer grâce aux données et aux modèles, puis de les amener
à réfléchir aux implications organisationnelles et sociétales de leurs
solutions. L'orientation pratique et partenariale de l'École de gestion
correspond très bien à cette approche.

Enfin, je suis particulièrement attiré par la culture de collaboration
interdisciplinaire et par l'accent mis sur la formation de leaders
capables de diriger la transformation numérique s'appuyant sur l'IA. Je
serais heureux de m'investir pleinement dans les projets du SIMQG, du
CROI et de l'École de gestion, en recherche, en enseignement et en
service à la collectivité.

Je vous remercie de l'attention portée à ma candidature et me tiens à
votre disposition pour toute information complémentaire.

Veuillez agréer, Madame, Monsieur, l'expression de mes salutations
distinguées.

Carlos Denner dos Santos carlosdenner@gmail.com +1 (438) 836-4116
