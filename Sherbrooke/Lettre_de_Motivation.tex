\documentclass[11pt, a4paper]{article}

\usepackage[margin=1in]{geometry}
\usepackage[utf8]{inputenc}
\usepackage[french]{babel}
\usepackage{hyperref}
\usepackage{ebgaramond}

\hypersetup{
    colorlinks=true,
    linkcolor=black,
    urlcolor=blue,
    citecolor=black
}

\pagestyle{empty}
\parindent=0pt
\parskip=10pt

\begin{document}

\noindent Sherbrooke, le 17 novembre 2025

\medskip

\noindent\textbf{Objet :} Candidature au poste de professeure ou professeur en gestion de l'intelligence artificielle (offre 07823)

\medskip

\noindent Madame, Monsieur,

\medskip

Je vous soumets ma candidature au poste de professeure ou professeur en gestion de l'intelligence artificielle au Département des systèmes d'information et méthodes quantitatives de gestion (SIMQG).

Avec un doctorat en systèmes d'information et plus de vingt ans d'expérience à l'intersection de l'analytique, des systèmes d'information et de la stratégie, je combine recherche rigoureuse et expertise pratique en déploiement de systèmes d'IA. Mes publications récentes dans \textit{Ethics and Information Technology} et \textit{Government Information Quarterly} portent sur la gouvernance et la régulation de l'IA dans les organisations. Parallèlement, j'ai conçu et déployé des systèmes analytiques et d'IA dans divers secteurs (télécommunications, énergie, santé, économie sociale).

Mon programme de recherche s'articule autour de trois axes directement alignés avec les besoins du département : (1) la gouvernance et la gestion des risques des systèmes d'IA en contexte organisationnel ; (2) l'ingénierie, les opérations et la sécurité des systèmes d'IA (sécurité des modèles de langage de grande taille—LLM, hallucinations, systèmes agentiques) ; et (3) la performance et l'impact durable de l'IA. L'objectif est de produire des contributions théoriques et des outils pratiques directement utiles aux organisations partenaires.

En enseignement, j'ai une expérience significative à tous les cycles, avec une approche fondée sur des problèmes réels, l'outillage par les données et modèles, et la réflexion sur les implications organisationnelles et sociétales. Cette pédagogie correspond très bien à l'orientation pratique et partenariale de l'École de gestion.

Je suis particulièrement attiré par la culture de collaboration interdisciplinaire du département et par sa mission de former des leaders capables de piloter la transformation numérique. Je serais heureux de contribuer pleinement aux projets du SIMQG, du Centre de recherche Createch sur les organisations intelligentes (CROI) et de l'École de gestion, en recherche, en enseignement et en service à la collectivité.

Je vous remercie de l'attention portée à ma candidature et me tiens à votre disposition pour toute information complémentaire.

Veuillez agréer, Madame, Monsieur, l'expression de mes salutations distinguées.

\vspace{16pt}

\noindent Carlos Denner dos Santos

\noindent\href{mailto:carlosdenner@gmail.com}{carlosdenner@gmail.com} \quad +1 (438) 836-4116

\end{document}
