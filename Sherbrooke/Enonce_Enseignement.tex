\documentclass[11pt, a4paper]{article}

\usepackage[margin=1in]{geometry}
\usepackage[utf8]{inputenc}
\usepackage[french]{babel}
\usepackage{hyperref}
\usepackage{ebgaramond}
\usepackage{parskip}

\hypersetup{
    colorlinks=true,
    linkcolor=black,
    urlcolor=blue,
    citecolor=black
}

\pagestyle{empty}
\setlength{\parindent}{0pt}

\title{Énoncé d'Enseignement\\{\normalsize Philosophie, Intentions et Intégration aux Efforts du Département}}
\author{Carlos Denner dos Santos}
\date{Novembre 2025}

\begin{document}

\maketitle

\vspace{-12pt}
\hrule
\vspace{12pt}

\section*{Philosophie d'Enseignement et Vision Pédagogique}

Je me vois d'abord comme quelqu'un qui aide les étudiantes et étudiants à utiliser les données, les systèmes d'information et l'intelligence artificielle pour résoudre des problèmes réels. Tout le reste --- concepts, méthodes, outils --- est au service de cet objectif.

Ma philosophie d'enseignement repose sur quatre principes fondamentaux :

\textbf{1) Ancrer l'apprentissage dans des situations authentiques.} Je pars de problèmes réels : un projet d'IA mal cadré, une organisation publique qui peine à gouverner ses données, une startup qui doit choisir un modèle d'affaires. Cette approche par problèmes m'a guidé dans mes cours de systèmes d'information, de gouvernance de l'IA et d'analytique, ainsi que dans ma supervision d'étudiants.

\textbf{2) Combiner rigueur conceptuelle et ``learning by doing''.} Je tiens beaucoup à l'articulation entre théorie et pratique. Les étudiants doivent comprendre les modèles et cadres, mais aussi les mettre en œuvre, les tester, les adapter. Un concept important est presque toujours associé à un exercice appliqué : une étude de cas, une mini-enquête, une analyse de données, ou un prototype. Cela a inspiré mes publications sur l'usage d'exercices en classe et de projets itératifs.

\textbf{3) Cultiver l'autonomie réflexive.} L'université est un lieu où l'on apprend à poser de bonnes questions. Je pousse les étudiants à réfléchir : pourquoi ce modèle plutôt qu'un autre ? Quelles hypothèses implicites ? Quelles implications éthiques ? Cette réflexivité est centrale pour la gouvernance de l'IA, les systèmes intelligents et l'analytique.

\textbf{4) Construire un climat inclusif et exigeant.} Je crée un environnement où les étudiants se sentent en sécurité pour poser des questions et sont mis au défi de sortir de leur zone de confort. Cela implique des attentes claires, un feedback régulier et transparent, et une attention aux différentes trajectoires. Ayant enseigné au Brésil, au Canada, en présentiel et en ligne, je suis particulièrement sensible à ces écarts.

\section*{Expérience d'Enseignement et Dispositifs Pédagogiques}

Au fil de ma carrière, j'ai enseigné à tous les cycles dans des programmes de gestion, systèmes d'information, data science et informatique appliquée. Mes champs principaux incluent : systèmes d'information et transformation numérique; gouvernance de l'IA, risques technologiques et sécurité des LLM; analytique d'affaires et science des données; programmation (Python) et systèmes RAG; entrepreneuriat et innovation.

Concrètement, mes cours s'organisent autour de dispositifs clés :

\textbf{Études de cas et projets appliqués.} Je fais un usage intensif de cas --- tirés de la littérature ou construits à partir de mes expériences de recherche et consultation --- pour analyser des situations réalistes et proposer des plans d'action. Je conçois aussi des projets de session où les équipes travaillent sur un problème réel (organisation partenaire, données ouvertes), avec livrables intermédiaires et restitution finale structurée.

\textbf{Laboratoires et studios de données et d'IA.} Pour les cours en programmation, analytique ou IA, je privilégie les séances où les étudiants manipulent eux-mêmes les données, modèles et outils (notebooks, plateformes analytiques, environnements MLOps/LLMOps simplifiés). L'idée est de passer rapidement du ``voir faire'' au ``faire soi-même''.

\textbf{Intégration des enjeux de gouvernance, d'éthique et de sécurité.} Dans mes cours sur l'IA, les systèmes d'information et l'analytique, j'intègre systématiquement les questions de gouvernance, d'éthique, de biais, de transparence, de responsabilité et de sécurité des LLM. Cela passe par des analyses de politiques et réglementations, des discussions sur des incidents réels (prompt injection, hallucinations, boucles affectives problématiques), et des exercices où les étudiants doivent proposer des mécanismes de contrôle et des structures de gouvernance.

\textbf{Usage réfléchi des outils numériques et des LLM.} Je considère les LLM comme des objets d'étude et de pratique. Je les utilise comme ``co-pilotes'' tout en expliquant leurs limites (hallucinations, biais, risques de sécurité). Cela développe chez les étudiants une posture de maîtrise critique plutôt qu'une adoption naïve.

\section*{Évaluation et Encadrement}

L'évaluation est d'abord un problème de conception : que mesurer vraiment, et comment le rendre transparent ? Je combine travaux d'équipe (analyses de cas, projets, prototypes), évaluations individuelles (quiz, mini-essais, journaux réflexifs, examens basés sur la résolution de problèmes) et coévaluation structurée.

En plus de l'enseignement formel, j'ai encadré de nombreux travaux de fin d'études, mémoires et projets d'amélioration en organisation, notamment en gouvernance des TI, transparence publique, gestion des incidents et transformation numérique. Ces encadrements se caractérisent par un cadrage solide, un accompagnement méthodologique et une exigence de retombées réelles pour les organisations partenaires.

\section*{Intégration aux Efforts du Département}

L'environnement de l'École de gestion et du SIMQG, fortement connecté aux milieux de pratique et à l'innovation pédagogique, est particulièrement propice pour poursuivre et amplifier ce travail.

\textbf{Contributions directes aux programmes.} Je peux co-développer ou actualiser des cours sur la gestion stratégique de l'IA, la gouvernance des systèmes d'IA, l'analytique d'affaires et la transformation numérique, en les ancrant dans des cas et projets issus de mes recherches et collaborations.

\textbf{Innovation pédagogique.} Je souhaite participer à des initiatives innovantes : développement de jeux sérieux ou simulations autour de la gouvernance de l'IA, utilisation de plateformes d'analytique et de LLM en contexte de cours, création de ressources pédagogiques bilingues (français/anglais) sur la gouvernance, la gestion des risques et l'ingénierie des systèmes d'IA.

\textbf{Supervision et développement d'expertise.} À Sherbrooke, je souhaite contribuer activement à la supervision d'étudiants à la maîtrise et au doctorat, ainsi qu'aux projets du microprogramme de 3e cycle en gestion stratégique de l'IA. Mon objectif est de former des personnes capables de parler le langage des dirigeants et celui des spécialistes techniques --- un profil clé pour la gestion de l'IA.

\section*{Conclusion}

Mon approche de l'enseignement est profondément liée à ma recherche et à mes expériences sur le terrain. Je cherche à outiller les étudiants pour concevoir, gouverner et critiquer les systèmes d'IA et de données dans des organisations réelles, avec rigueur conceptuelle et sens pratique. Je serais heureux de poursuivre et d'amplifier cette démarche au sein de l'École de gestion de l'Université de Sherbrooke.

\end{document}
